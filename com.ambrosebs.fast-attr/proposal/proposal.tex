\documentclass[10pt]{article}

\usepackage{amsmath}
\usepackage[backend=bibtex]{biblatex}

%\setlength{\parindent}{4em}
%\setlength{\parskip}{1em}
%\renewcommand{\baselinestretch}{1.5}

\addbibresource{bibliography.bib}

\usepackage[utf8]{inputenc}
\usepackage[english]{babel}
\usepackage{enumitem}

\begin{document}
\title{Optimized Keyword Persistent Hash Maps for Clojure
\\
\large
B503 - Project Proposal}
\author{Ambrose Bonnaire-Sergeant (0003410123)}

\maketitle

\section*{Background}

Clojure provides a suite of persistent data structures,
including persistent hash maps.
The most common use of hash maps are small maps with keyword keys
(which we refer to as ``keyword maps'').
They are so important that Typed Clojure, an optional type
system for Clojure, implements heterogeneous map types
to represent such keyword maps.

The current hash map implementation is implemented by Hickey
(described in detail by L'orange~\cite{HyperionBlog1})
,
based on previous work by Bagwell~\cite{bagwell2001ideal}
It uses Hash Array Mapped Tries (HAMT) as its basis.
It is not optimized for keywords.

I plan to investigate different optimizations to specifically
target small maps.

\section*{Deliverables}

In my paper, I will:

\begin{enumerate}
\item summarise the history of HAMT,
\item discuss various characteristics of HAMT's,
\item benchmark HMAT's against other hash maps runnable from the JVM
\item Reimplement the Clojure HAMT implementation in Clojure (originally in Java)
\item describe how this implementation works
\item attempt Clojure-specific optimisations and benchmark against the original implementation.
\item analyze why performance characterists change,
\item speculate on future work/directions for HAMT's.
\end{enumerate}

\printbibliography
\end{document}
